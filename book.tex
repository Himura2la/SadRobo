%%%%%%%%%%%%%%%%%%%%%%%%%%%%%%%%%%%%%%%%%
% SadRobo Comicbook
% Version 0.1 (08/05/16)
%
% This template has been downloaded from:
% https://github.com/Himura2la/SadRobo
%
% Original author:
% SadBobo Team <https://new.vk.com/sadrobo>
%
% License:
% Sad License 
%
% Важные замечания:
% Роботу всё ещё грустно.
%%%%%%%%%%%%%%%%%%%%%%%%%%%%%%%%%%%%%%%%%

\documentclass[17pt]{memoir} % 14, 17, 20
% !TeX root = book.tex

% ------- Пакеты -------

\usepackage{cmap} % Для кодировки PDF
\usepackage[utf8]{inputenc} % Кодировка входа
\usepackage[encoding,filenameencoding=utf8]{grffile} %Нормально читаем кодировку файловой системы
\usepackage[T2A]{fontenc} % Кодировка выхода
\usepackage[russian]{babel} % Локаль
\usepackage{pscyr} % Крутые русские шрифты (http://scon155.phys.msu.su/~swan/cyrtug2000.pdf)
\usepackage[babel=true]{microtype} % Точный подгон текста
\usepackage{graphicx} % Картинки
\usepackage[unicode=true,bookmarks=true,bookmarksnumbered=false,bookmarksopen=false,breaklinks=false,pdfborder={0 0 1},backref=section,colorlinks=false]{hyperref} % Мета-информация для pdf


% ------- Пути к файлам -------

\makeatletter
	\def\input@path{{texts/}}           % К текстовым файлам (для input)
\makeatother

\graphicspath{{pictures/}}              % К картинкам
\DeclareGraphicsExtensions{.png,.jpg}   % Расширения для используемых картинок

% ------- Геометрия -------

\setstocksize{297mm}{210mm} % Размер исходников
\settrimmedsize{200mm}{185mm}{*} % Размер обрезанной страницы 
\settrims{5mm}{2cm} 			 % Размер обреза сверху/изнутри
% Max 10:43 am
% с 3 сторон по 5 мм с крепления 2 см
\setlrmarginsandblock{10mm}{10mm}{*} % Отступ слева/справа
\setulmarginsandblock{10mm}{15mm}{*} % Отступ сверху/снизу
\setheadfoot{0pt}{1.5em} % Высота шапки и подвала
\setheaderspaces{*}{0pt}{*} % Пространство между шапкой и блоком текста
\setmarginnotes{0pt}{0pt}{0pt} % Ликвидация области заметок

%\usepackage{showframe} % Любые манипуляции с полями проводить с этим пакетом


% ------- Колонтитулы -------

\setlength{\headwidth}{\textwidth} % Ширина шапки, как и у текста
\makeheadrule{plain}{0pt}{0pt} % No header line
\makeoddhead{plain}{
	%\scriptsize{\theauthor\hskip.2cm\vrule\hskip.2cm\itshape{\thetitle}}
}{}{} % Спецификация нечетной шапки (текст)
\makeevenhead{plain}{
	%\scriptsize{\theauthor\hskip.2cm\vrule\hskip.2cm\itshape{\thetitle}}
}{}{} % Спецификация четной шапки (пикча)


\newcounter{dianum}
\setcounter{dianum}{5}
\makeoddfoot{plain}{}{
	{\arabic{dianum}}
}{} % Спецификация нечетного подвала (текст)
\makeevenfoot{plain}{}{
	{\arabic{dianum}}
}{} % Спецификация четного подвала (пикча)
\pagestyle{plain}


% ------- Оформление названия главы -------

\makechapterstyle{RoboChapter}{ % Определяем новый стиль
	\renewcommand{\chapterheadstart}{} % Пробел перед началом главы
	\renewcommand{\printchaptername}{} % Спецификация слова "Глава"
	\renewcommand{\printchapternum}{} % Спецификация номера главы
	\renewcommand{\afterchapternum}{} % Пробел между номером главы и текстом
	\renewcommand{\printchaptertitle}[1]{ % Спецификация названия главы
	\centering \itshape\Huge{##1}}
	\renewcommand{\afterchaptertitle}{
		\vskip0.5\onelineskip        %Отступ после названия главы
	}}
\chapterstyle{RoboChapter} % Используем новый стиль


% ------- Еще немного настроек -------

\setlength{\parindent}{0pt} % No paragraph indentation!
\midsloppy % Перенос строк, что-то между \fussy и \sloppy
\checkandfixthelayout % just do it, memoir
\renewcommand{\familydefault}{far} % Шрифт (tex\latex\pscyr\pscyr.sty)   % Подключаем чудо-шрифты из pscyr % Загружаем файл с настройками


% ------- Костыль для кириллицы в input -------

\makeatletter  
    \let\old@input\input
    \renewcommand\input[1]{
        \expandafter\old@input{"\detokenize{#1}"}
    }
\makeatother


% ------- Считаем оптимальную ширину картинки  -------
% ------- Чтобы макрос работал как нужно, не должны быть указаны резолюшны. Долой жепеге!

\makeatletter 
\def \maxwidth{
	\ifdim\Gin@nat@width>159.5mm
		159.5mm
	\else
		\Gin@nat@width
	\fi}
\makeatother

% ------- Служебные переменные для длины и ширины картинки ------- 
\newlength\imageheight
\newlength\imagewidth


% ------- Робо-макросы ------- 

\newcommand\R{\par\noindent---~}  % Маркер робота
\newcommand\X{\par\noindent---~}  % Маркер собеседника
\newcommand\Y{\par\noindent---~}  % Маркер собеседника 1
\newcommand\Z{\par\noindent---~}  % Маркер собеседника 2

\newenvironment{dialog} % Пространство диалога
{  } % Выполняется в начале
{  } % Выполняется в конце

\newenvironment{monolog} % Пространство размышлений
{ \begin{quote}\itshape } % Выполняется в начале
{ \end{quote} } % Выполняется в конце

\newcommand{\newdialog}[3]{
	\chapter{Робот, которому грустно, и #2}
		\input{#2}
		\settoheight\imageheight{\includegraphics{#2}}
		\settowidth\imagewidth{\includegraphics{#2}}
		\ifthenelse{\imagewidth > \imageheight}{
			\begin{figure} %Горизонтальная картинка
				\centering
				\mbox{   %Эта фигня до теста, если её убрать, то картинка будет размером как рамка
					\includegraphics[width = \maxwidth]{#2}    %Прописал числами, а то что-расходятся все параметры
				}			
				\centering #3
			\end{figure}
		}{
			\begin{figure}  % Вертикальная картинка
				\begin{tabu} to 1\textwidth { X[c,m] X[0.1,c,m] }
					\includegraphics[height=\textheight]{#2} &
					\rotatebox{90}{#3}
				\end{tabu}
			\end{figure}
		}
	\setcounter{dianum}{#1}
}



\begin{document} % --------------------------------

\newdialog{002}{Викинги}{Борода лучше своего отсутствия}
\newdialog{004}{Создатель}{Все мы что"~то создаём}
\newdialog{014}{Паук}{Тащемта, например}
\newdialog{018}{Ницше}{Сверхчеловек не нашего времени}
\newdialog{029}{Все}{Это можно было бы назвать вечеринкой, если Вечер бы пришёл. Но пришёл только Все.}
\newdialog{035}{Студенты}{Фанатичные адепты богини Халявы}
\newdialog{036}{Митинг}{Листовки, шарики и атрибутика в подарок!}
\newdialog{040}{Мой здравый смысл}{Вернись ко мне, когда наступит вчера}
\newdialog{061}{Утренняя газета}{Все"~равно её никто не читает}
\newdialog{071}{Лао Цзы}{Восточная философия~--- сложна и опасна}
\newdialog{072}{Время}{Его всегда слишком много, но никогда не хватает}
\newdialog{074}{Даос-какаос}{Ох это прекрасное, освежающее и бодрящее столкновение с посохом} %BAD: Дефис в заголовке
\newdialog{088}{Мотивация}{Тут что"~то нужно ещё добавить?}
\newdialog{089}{Политика}{Нельзя коснуться, не испачкавшись}
\newdialog{090}{Маркетинг}{Без него не было бы этой книги}
\newdialog{095}{Баян}{Где"~то я всё это уже видел}
\newdialog{118}{Калечные девы}{Датфил}
\newdialog{119}{Лекция}{Я там был, инфа 100\%}
\newdialog{144}{Шайа Лабаф}{Джазовый дуэт}
\newdialog{147}{Шон Бин}{Казалось бы, причём здесь Караваждо?}
\newdialog{150}{Аниме}{Примерно так я и стал хиккой}
\newdialog{158}{Председатель земного шара}{Когда умирают травы~--- сохнут}
\newdialog{12}{Живая изгородь}{Ожидая Вечность на пороге Бездны он познал Одиночество.}
\newdialog{17}{Бомжи-опенсорсники}{Если вы знаете о грядущих конференциях с бесплатной едой, обязательно свяжитесь с Авторы!} %BAD: Дефис в заголовке
\newdialog{23}{Бездна}{Если долго всматриваться в Бездну, то у вас слишком много свободного времени.}
\newdialog{30}{Ректор}{Иногда я оглядываюсь назад и вижу его. Жутко.}
\newdialog{46}{Буддийский монах}{Не забудьте вбить в гугл <<Цвань Лю Чжи>> }
\newdialog{47}{Онотоле}{\parbox{\textheight}{\centering Вечно увиливает от ответа на вопрос <<В чём смысл жизни?>>}}
\newdialog{48}{Всеведущий}{Он знал, что я это напишу}
\newdialog{49}{Монголы}{Богатырь в карауле когда"~то был есаулом }
\newdialog{76}{Солипсическая бочка}{Лучшее место в мире }
\newdialog{77}{Мэттью Макконахи}{\parbox{\textheight}{\centering Я шутил про Мэттью Макконахи ещё до того, как мне стали платить за это. С тех пор мало что изменилось.}}
\newdialog{103}{Смысл жизни}{Словно хлопок одной ладонью}
\newdialog{92}{Город}{Мы не несём ответственности за то, что кто"~то узнал здесь свой родной город}
\newdialog{93}{Философия}{Тот, кто знает о нас меньше, чем мы о нём}
\newdialog{96}{Актуальность моей работы}{Особенно Время отыгрался на нём}
\newdialog{100}{Зло}{Раньше им пугали детишек\ldotst{}}
\newdialog{101}{Добро}{А он должен был приходить и спасать\ldotst{}}
\newdialog{102}{Абсолютное зло}{А им пугали взрослых\ldotst{}}
\newdialog{103}{Абсолютное добро}{Но он был только в DnD\ldotst{}}
\newdialog{104}{Справедливость}{В задницу справедливость, хочу бэтмобиль!}
\newdialog{106}{Лоли}{А почему бы и нет?}
\newdialog{108}{Эльпедо Медведо}{Когда закончились машины}
\newdialog{160}{Уныние}{Даже эти слова я пишу с большой грустью внутри}


\end{document}