{\small \begin{dialog}
\X Робот. И работа в кювет. Привет.
\end{dialog}

\begin{monolog}
Возник и передом зачеловечил длинный чернотой сюртук, плюющийся стянутыми обмотками парусиновых брюк. Рассеянно улыбнувшись, он протянул руку обутую в изрешеченную щиблету.
\end{monolog}

\begin{dialog}
\R Привет, \#Председатель земного шара.
\X Поможешь мне найти гвозди в темноте?
\R Может, включить свет?
\X Как я тогда смогу искать гвозди в темноте?
\end{dialog}

\begin{monolog}
Робот нашёл такое замечание логичным, и решил помочь поэту. Комната была крохотной и совершенно пустой, а вместо гвоздей попадались лишь исписанные каракулями листочки.
\end{monolog}

\begin{dialog}
\R Это твои \#поэмы?
\X Да, я пишу их в темноте, когда не могу найти гвозди.
\R Тебе это удобно?
\X Нет, здесь слишком много мебели.
\end{dialog}

\begin{dialog}
\R \#Грустно.
\end{dialog}}